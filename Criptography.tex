%% LyX 2.1.2 created this file.  For more info, see http://www.lyx.org/.
%% Do not edit unless you really know what you are doing.
\documentclass[twocolumn,compsoc,journal]{IEEEtran}
\usepackage[T1]{fontenc}
\usepackage{calc}
\usepackage{amsthm}
\usepackage[unicode=true,
 bookmarks=true,bookmarksnumbered=true,bookmarksopen=true,bookmarksopenlevel=1,
 breaklinks=false,pdfborder={0 0 0},backref=false,colorlinks=false]
 {hyperref}
\hypersetup{pdftitle={Your Title},
 pdfauthor={Your Name},
 pdfpagelayout=OneColumn, pdfnewwindow=true, pdfstartview=XYZ, plainpages=false}

\makeatletter

%%%%%%%%%%%%%%%%%%%%%%%%%%%%%% LyX specific LaTeX commands.
\DeclareRobustCommand*{\lyxarrow}{%
\@ifstar
{\leavevmode\,$\triangleleft$\,\allowbreak}
{\leavevmode\,$\triangleright$\,\allowbreak}}
%% Because html converters don't know tabularnewline
\providecommand{\tabularnewline}{\\}

%%%%%%%%%%%%%%%%%%%%%%%%%%%%%% Textclass specific LaTeX commands.
 % protect \markboth against an old bug reintroduced in babel >= 3.8g
 \let\oldforeign@language\foreign@language
 \DeclareRobustCommand{\foreign@language}[1]{%
   \lowercase{\oldforeign@language{#1}}}
\theoremstyle{plain}
\newtheorem{thm}{\protect\theoremname}
\theoremstyle{plain}
\newtheorem{lem}[thm]{\protect\lemmaname}

%%%%%%%%%%%%%%%%%%%%%%%%%%%%%% User specified LaTeX commands.
% for subfigures/subtables
\usepackage[caption=false,font=normalsize,labelfont=sf,textfont=sf]{subfig}
%\usepackage[nocompress]{cite} %optional

\makeatother

\providecommand{\lemmaname}{Lemma}
\providecommand{\theoremname}{Theorem}

\begin{document}





\title{Cyber Attacks and RSA Integer Factorization Attacks}


\author{Oscar~Dussan,~David~Rodas,~Jessid~Mejia,~and~Camilo~Gonzalez}


\markboth{Introduction to Crytography}{}

\IEEEtitleabstractindextext{
\begin{abstract}
Through this document we will try to explain what it is a Cyber-Attack,
also we will define some types and we will use RSA to show some examples
of how they work.
\end{abstract}

}

\maketitle

\IEEEdisplaynontitleabstractindextext{}


\IEEEpeerreviewmaketitle{}


\section{Introduction}

\IEEEPARstart{C}{}ryptography is about constructing and analyzing
protocols that prevent third parties or the public from reading private
messages, in that matter what we are trying to do is to identify that
while there is always someone with the intention of protecting their
information and then communicating it to specific people, there is
a third party with the interest of finding this information. Since
there is some system to hide this information there will be someone
trying to break it. Cyber-Attacks are all these methods designed to
break security. 


\section{What is a Cyber-Attack?}

Is any type of offensive maneuver employed by individuals or whole
organizations that targets computer information systems, infrastructures,
computer networks, and/or personal computer devices by various means
of malicious acts usually originating from an anonymous source that
either steals, alters or destroys information. Cyber-attacks can range
from installing spyware on a system to DDoS that can bring down entire
servers.


\section{Types of Cyber-Attacks. }


\section{Previous Work}

text text text text text text text text text text text text text text
text


\subsection{subsection}


\subsection{another subsection}


\section{Methodology}
\begin{thm}[Theorem name]
For a named theorem or theorem-like environment you need to insert
the name through \textsf{Insert\lyxarrow{}Short Title}, as done here.\end{thm}
\begin{lem}
If you don't want a theorem or lemma name don't add one.\end{lem}
\begin{IEEEproof}
And here's the proof!
\end{IEEEproof}

\section{Results}

\begin{figure}[htbp]
\begin{centering}
\textsf{A single column figure goes here}
\par\end{centering}

\protect\caption{Captions go \emph{under} the figure}
\end{figure}
\begin{table}[htbp]
\protect\caption{Table captions go \emph{above} the table}


\centering{}%
\begin{tabular}{|c|c|}
\hline 
delete & this\tabularnewline
\hline 
\hline 
example & table\tabularnewline
\hline 
\end{tabular}
\end{table}



\section{Conclusions}

bla bla


\appendices{}


\section{First appendix}

Citation: \cite{IEEEexample:beebe_archive}


\section{Second appendix}


\section*{Acknowlegment}

bla bla



\bibliographystyle{IEEEtran}
\bibliography{IEEEabrv,IEEEexample}

\begin{IEEEbiography}[{\fbox{\begin{minipage}[t][1.25in]{1in}%
Replace this box by an image with a width of 1\,in and a height of
1.25\,in!%
\end{minipage}}}]{Your Name}
 All about you and the what your interests are.
\end{IEEEbiography}


\begin{IEEEbiographynophoto}{Coauthor}
Same again for the co-author, but without photo\end{IEEEbiographynophoto}


\end{document}
